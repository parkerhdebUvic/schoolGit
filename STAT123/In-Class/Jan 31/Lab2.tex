% Options for packages loaded elsewhere
\PassOptionsToPackage{unicode}{hyperref}
\PassOptionsToPackage{hyphens}{url}
%
\documentclass[
]{article}
\usepackage{amsmath,amssymb}
\usepackage{lmodern}
\usepackage{iftex}
\ifPDFTeX
  \usepackage[T1]{fontenc}
  \usepackage[utf8]{inputenc}
  \usepackage{textcomp} % provide euro and other symbols
\else % if luatex or xetex
  \usepackage{unicode-math}
  \defaultfontfeatures{Scale=MatchLowercase}
  \defaultfontfeatures[\rmfamily]{Ligatures=TeX,Scale=1}
\fi
% Use upquote if available, for straight quotes in verbatim environments
\IfFileExists{upquote.sty}{\usepackage{upquote}}{}
\IfFileExists{microtype.sty}{% use microtype if available
  \usepackage[]{microtype}
  \UseMicrotypeSet[protrusion]{basicmath} % disable protrusion for tt fonts
}{}
\makeatletter
\@ifundefined{KOMAClassName}{% if non-KOMA class
  \IfFileExists{parskip.sty}{%
    \usepackage{parskip}
  }{% else
    \setlength{\parindent}{0pt}
    \setlength{\parskip}{6pt plus 2pt minus 1pt}}
}{% if KOMA class
  \KOMAoptions{parskip=half}}
\makeatother
\usepackage{xcolor}
\IfFileExists{xurl.sty}{\usepackage{xurl}}{} % add URL line breaks if available
\IfFileExists{bookmark.sty}{\usepackage{bookmark}}{\usepackage{hyperref}}
\hypersetup{
  pdftitle={Lab2},
  pdfauthor={Parker DeBruyne - V00837207},
  hidelinks,
  pdfcreator={LaTeX via pandoc}}
\urlstyle{same} % disable monospaced font for URLs
\usepackage[margin=1in]{geometry}
\usepackage{graphicx}
\makeatletter
\def\maxwidth{\ifdim\Gin@nat@width>\linewidth\linewidth\else\Gin@nat@width\fi}
\def\maxheight{\ifdim\Gin@nat@height>\textheight\textheight\else\Gin@nat@height\fi}
\makeatother
% Scale images if necessary, so that they will not overflow the page
% margins by default, and it is still possible to overwrite the defaults
% using explicit options in \includegraphics[width, height, ...]{}
\setkeys{Gin}{width=\maxwidth,height=\maxheight,keepaspectratio}
% Set default figure placement to htbp
\makeatletter
\def\fps@figure{htbp}
\makeatother
\setlength{\emergencystretch}{3em} % prevent overfull lines
\providecommand{\tightlist}{%
  \setlength{\itemsep}{0pt}\setlength{\parskip}{0pt}}
\setcounter{secnumdepth}{-\maxdimen} % remove section numbering
\ifLuaTeX
  \usepackage{selnolig}  % disable illegal ligatures
\fi

\title{Lab2}
\author{Parker DeBruyne - V00837207}
\date{02/02/2022}

\begin{document}
\maketitle

\#Lab 2: Matrices The following worksheet is due by 8pm ONE day after
your lab. You can find the submission dropbox in Brightspace by clicking
on Content − \textgreater{} Lab Content.

\hypertarget{open-a-new-r-markdown-file.}{%
\subsection{0. Open a new R Markdown
file.}\label{open-a-new-r-markdown-file.}}

Note: Your worksheet is to be submitted as the output of an R Markdown
file (you can knit it to HTML and then convert it to PDF, or you can
knit it to PDF if you have LaTeX on your computer, or you can knit it to
Word and then convert that to a PDF).

\hypertarget{download-the-data-set-flowerdata-posted-under-lab-content-in-brightspace-under-lab-2-and-save-it-to-whatever-file-you-are-using-for-this-course.}{%
\subsection{1. Download the data set FlowerData posted under Lab Content
in Brightspace (under Lab 2) and save it to whatever file you are using
for this
course.}\label{download-the-data-set-flowerdata-posted-under-lab-content-in-brightspace-under-lab-2-and-save-it-to-whatever-file-you-are-using-for-this-course.}}

\hypertarget{a-read-the-flowerdata-file-into-r-and-call-it-f-data.-b-is-f-data-a-matrix-or-a-data-frame-explain-why.}{%
\subsubsection{(a) Read the FlowerData file into R and call it F data.
(b) Is F data a matrix or a data frame? Explain
why.}\label{a-read-the-flowerdata-file-into-r-and-call-it-f-data.-b-is-f-data-a-matrix-or-a-data-frame-explain-why.}}

\hypertarget{c-create-a-matrix-called-flowermatrix-which-contains-the-numerical-columns-of-fdata.-d-re-name-the-columns-of-flowermatrix-to-be-age-in-days-and-height-in-cm.}{%
\subsubsection{(c) Create a matrix called FlowerMatrix which contains
the numerical columns of Fdata. (d) Re-name the columns of FlowerMatrix
to be: Age (in days) and Height (in
cm).}\label{c-create-a-matrix-called-flowermatrix-which-contains-the-numerical-columns-of-fdata.-d-re-name-the-columns-of-flowermatrix-to-be-age-in-days-and-height-in-cm.}}

\hypertarget{e-re-name-the-rows-of-flowermatrix-using-the-individuals-column-from-fdata.}{%
\subsubsection{(e) Re-name the rows of FlowerMatrix using the
Individuals column from
Fdata.}\label{e-re-name-the-rows-of-flowermatrix-using-the-individuals-column-from-fdata.}}

\hypertarget{section}{%
\subsection{2.}\label{section}}

\hypertarget{a-determine-the-average-age-of-the-flowers-in-the-data-set.-b-determine-the-average-height-of-the-flowers-in-the-data-set.}{%
\subsubsection{(a) Determine the average age of the flowers in the data
set. (b) Determine the average height of the flowers in the data
set.}\label{a-determine-the-average-age-of-the-flowers-in-the-data-set.-b-determine-the-average-height-of-the-flowers-in-the-data-set.}}

\hypertarget{c-determine-which-individual-has-the-largest-height-and-which-individual-is-the-oldest-flower.}{%
\subsubsection{(c) Determine which individual has the largest height and
which individual is the oldest
flower.}\label{c-determine-which-individual-has-the-largest-height-and-which-individual-is-the-oldest-flower.}}

\hypertarget{d-determine-which-individual-has-the-smallest-height-and-which-individual-is-the-youngest-flower.}{%
\subsubsection{(d) Determine which individual has the smallest height
and which individual is the youngest
flower.}\label{d-determine-which-individual-has-the-smallest-height-and-which-individual-is-the-youngest-flower.}}

\hypertarget{e-what-colour-are-the-flowers-in-your-answer-to-parts-c-and-d}{%
\subsubsection{(e) What colour are the flowers in your answer to parts
(c) and
(d)?}\label{e-what-colour-are-the-flowers-in-your-answer-to-parts-c-and-d}}

\hypertarget{section-1}{%
\subsection{3.}\label{section-1}}

\hypertarget{a-how-many-rows-are-there-in-the-matrix-call-this-value-n.}{%
\subsubsection{(a) How many rows are there in the matrix? Call this
value
n.}\label{a-how-many-rows-are-there-in-the-matrix-call-this-value-n.}}

\hypertarget{b-take-a-random-sample-of-size-15-between-the-numbers-1-to-n.-call-this-sample-samp.}{%
\subsubsection{(b) Take a random sample of size 15 between the numbers 1
to n.~Call this sample
samp.}\label{b-take-a-random-sample-of-size-15-between-the-numbers-1-to-n.-call-this-sample-samp.}}

\hypertarget{c-create-a-new-matrix-called-samplematrix-that-contains-the-rows-samp-from-flowermatrix.-for-example-if-your-sample-came-back-as-numbers-1-15-which-is-very-unlikely-then}{%
\subsubsection{(c) Create a new matrix called SampleMatrix that contains
the rows samp from FlowerMatrix. For example, if your sample came back
as numbers 1-15 (which is very unlikely),
then}\label{c-create-a-new-matrix-called-samplematrix-that-contains-the-rows-samp-from-flowermatrix.-for-example-if-your-sample-came-back-as-numbers-1-15-which-is-very-unlikely-then}}

your SampleMatrix would contain the first 15 rows of FlowerMatrix.

\hypertarget{d-what-is-the-sample-average-height-and-age-of-the-flowers-in-your-samplematrix-are-they-similar-to-your-answer-to-q2.-a-and-b}{%
\subsubsection{(d) What is the sample average height and age of the
flowers in your SampleMatrix? Are they similar to your answer to Q2. (a)
and
(b)?}\label{d-what-is-the-sample-average-height-and-age-of-the-flowers-in-your-samplematrix-are-they-similar-to-your-answer-to-q2.-a-and-b}}

\end{document}
