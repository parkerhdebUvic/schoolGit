% Options for packages loaded elsewhere
\PassOptionsToPackage{unicode}{hyperref}
\PassOptionsToPackage{hyphens}{url}
%
\documentclass[
]{article}
\usepackage{amsmath,amssymb}
\usepackage{lmodern}
\usepackage{iftex}
\ifPDFTeX
  \usepackage[T1]{fontenc}
  \usepackage[utf8]{inputenc}
  \usepackage{textcomp} % provide euro and other symbols
\else % if luatex or xetex
  \usepackage{unicode-math}
  \defaultfontfeatures{Scale=MatchLowercase}
  \defaultfontfeatures[\rmfamily]{Ligatures=TeX,Scale=1}
\fi
% Use upquote if available, for straight quotes in verbatim environments
\IfFileExists{upquote.sty}{\usepackage{upquote}}{}
\IfFileExists{microtype.sty}{% use microtype if available
  \usepackage[]{microtype}
  \UseMicrotypeSet[protrusion]{basicmath} % disable protrusion for tt fonts
}{}
\usepackage{xcolor}
\usepackage[margin=1in]{geometry}
\usepackage{color}
\usepackage{fancyvrb}
\newcommand{\VerbBar}{|}
\newcommand{\VERB}{\Verb[commandchars=\\\{\}]}
\DefineVerbatimEnvironment{Highlighting}{Verbatim}{commandchars=\\\{\}}
% Add ',fontsize=\small' for more characters per line
\usepackage{framed}
\definecolor{shadecolor}{RGB}{248,248,248}
\newenvironment{Shaded}{\begin{snugshade}}{\end{snugshade}}
\newcommand{\AlertTok}[1]{\textcolor[rgb]{0.94,0.16,0.16}{#1}}
\newcommand{\AnnotationTok}[1]{\textcolor[rgb]{0.56,0.35,0.01}{\textbf{\textit{#1}}}}
\newcommand{\AttributeTok}[1]{\textcolor[rgb]{0.77,0.63,0.00}{#1}}
\newcommand{\BaseNTok}[1]{\textcolor[rgb]{0.00,0.00,0.81}{#1}}
\newcommand{\BuiltInTok}[1]{#1}
\newcommand{\CharTok}[1]{\textcolor[rgb]{0.31,0.60,0.02}{#1}}
\newcommand{\CommentTok}[1]{\textcolor[rgb]{0.56,0.35,0.01}{\textit{#1}}}
\newcommand{\CommentVarTok}[1]{\textcolor[rgb]{0.56,0.35,0.01}{\textbf{\textit{#1}}}}
\newcommand{\ConstantTok}[1]{\textcolor[rgb]{0.00,0.00,0.00}{#1}}
\newcommand{\ControlFlowTok}[1]{\textcolor[rgb]{0.13,0.29,0.53}{\textbf{#1}}}
\newcommand{\DataTypeTok}[1]{\textcolor[rgb]{0.13,0.29,0.53}{#1}}
\newcommand{\DecValTok}[1]{\textcolor[rgb]{0.00,0.00,0.81}{#1}}
\newcommand{\DocumentationTok}[1]{\textcolor[rgb]{0.56,0.35,0.01}{\textbf{\textit{#1}}}}
\newcommand{\ErrorTok}[1]{\textcolor[rgb]{0.64,0.00,0.00}{\textbf{#1}}}
\newcommand{\ExtensionTok}[1]{#1}
\newcommand{\FloatTok}[1]{\textcolor[rgb]{0.00,0.00,0.81}{#1}}
\newcommand{\FunctionTok}[1]{\textcolor[rgb]{0.00,0.00,0.00}{#1}}
\newcommand{\ImportTok}[1]{#1}
\newcommand{\InformationTok}[1]{\textcolor[rgb]{0.56,0.35,0.01}{\textbf{\textit{#1}}}}
\newcommand{\KeywordTok}[1]{\textcolor[rgb]{0.13,0.29,0.53}{\textbf{#1}}}
\newcommand{\NormalTok}[1]{#1}
\newcommand{\OperatorTok}[1]{\textcolor[rgb]{0.81,0.36,0.00}{\textbf{#1}}}
\newcommand{\OtherTok}[1]{\textcolor[rgb]{0.56,0.35,0.01}{#1}}
\newcommand{\PreprocessorTok}[1]{\textcolor[rgb]{0.56,0.35,0.01}{\textit{#1}}}
\newcommand{\RegionMarkerTok}[1]{#1}
\newcommand{\SpecialCharTok}[1]{\textcolor[rgb]{0.00,0.00,0.00}{#1}}
\newcommand{\SpecialStringTok}[1]{\textcolor[rgb]{0.31,0.60,0.02}{#1}}
\newcommand{\StringTok}[1]{\textcolor[rgb]{0.31,0.60,0.02}{#1}}
\newcommand{\VariableTok}[1]{\textcolor[rgb]{0.00,0.00,0.00}{#1}}
\newcommand{\VerbatimStringTok}[1]{\textcolor[rgb]{0.31,0.60,0.02}{#1}}
\newcommand{\WarningTok}[1]{\textcolor[rgb]{0.56,0.35,0.01}{\textbf{\textit{#1}}}}
\usepackage{graphicx}
\makeatletter
\def\maxwidth{\ifdim\Gin@nat@width>\linewidth\linewidth\else\Gin@nat@width\fi}
\def\maxheight{\ifdim\Gin@nat@height>\textheight\textheight\else\Gin@nat@height\fi}
\makeatother
% Scale images if necessary, so that they will not overflow the page
% margins by default, and it is still possible to overwrite the defaults
% using explicit options in \includegraphics[width, height, ...]{}
\setkeys{Gin}{width=\maxwidth,height=\maxheight,keepaspectratio}
% Set default figure placement to htbp
\makeatletter
\def\fps@figure{htbp}
\makeatother
\setlength{\emergencystretch}{3em} % prevent overfull lines
\providecommand{\tightlist}{%
  \setlength{\itemsep}{0pt}\setlength{\parskip}{0pt}}
\setcounter{secnumdepth}{-\maxdimen} % remove section numbering
\ifLuaTeX
  \usepackage{selnolig}  % disable illegal ligatures
\fi
\IfFileExists{bookmark.sty}{\usepackage{bookmark}}{\usepackage{hyperref}}
\IfFileExists{xurl.sty}{\usepackage{xurl}}{} % add URL line breaks if available
\urlstyle{same} % disable monospaced font for URLs
\hypersetup{
  pdftitle={STAT 260 Fall 2023: R Assignment 2},
  pdfauthor={Parker DeBruyne - V00837207},
  hidelinks,
  pdfcreator={LaTeX via pandoc}}

\title{STAT 260 Fall 2023: R Assignment 2}
\author{Parker DeBruyne - V00837207}
\date{13/03/2023}

\begin{document}
\maketitle

\hypertarget{part-1}{%
\subsection{Part 1}\label{part-1}}

A radioactive object emits particles according to a Poisson process at
an average rate of 5.5 particles per second. We observe the object for a
total of 6.5 seconds. \newline \newline \textbf{(a) {[}1 mark{]} What is
the probability that no more than 40 particles will be emitted during
this interval?}

\begin{Shaded}
\begin{Highlighting}[]
\NormalTok{result }\OtherTok{=} \FunctionTok{ppois}\NormalTok{(}\DecValTok{40}\NormalTok{, }\AttributeTok{lambda=}\NormalTok{(}\FloatTok{5.5}\SpecialCharTok{*}\FloatTok{6.5}\NormalTok{))}

\NormalTok{result}
\end{Highlighting}
\end{Shaded}

\begin{verbatim}
## [1] 0.7896234
\end{verbatim}

\begin{Shaded}
\begin{Highlighting}[]
\NormalTok{output }\OtherTok{=} \FunctionTok{paste}\NormalTok{(}\StringTok{" Probability that no more than 40 }\SpecialCharTok{\textbackslash{}n}\StringTok{"}\NormalTok{,}
               \StringTok{"particles will be emitted during }\SpecialCharTok{\textbackslash{}n}\StringTok{"}\NormalTok{, }
               \StringTok{"a 6.5 second interval:"}\NormalTok{,}
               \FunctionTok{round}\NormalTok{(}\DecValTok{100}\SpecialCharTok{*}\NormalTok{result, }\DecValTok{0}\NormalTok{), }\StringTok{"\%"}\NormalTok{)}
\FunctionTok{writeLines}\NormalTok{(output)}
\end{Highlighting}
\end{Shaded}

\begin{verbatim}
##  Probability that no more than 40 
##  particles will be emitted during 
##  a 6.5 second interval: 79 %
\end{verbatim}

\newline
\newline

\textbf{(b) {[}1 mark{]} What is the probability that exactly 38
particles will be emitted during this interval?}

\begin{Shaded}
\begin{Highlighting}[]
\NormalTok{result }\OtherTok{=} \FunctionTok{dpois}\NormalTok{(}\DecValTok{38}\NormalTok{, }\AttributeTok{lambda=}\NormalTok{(}\FloatTok{5.6}\SpecialCharTok{*}\FloatTok{6.5}\NormalTok{))}

\NormalTok{result}
\end{Highlighting}
\end{Shaded}

\begin{verbatim}
## [1] 0.06237535
\end{verbatim}

\begin{Shaded}
\begin{Highlighting}[]
\NormalTok{output }\OtherTok{=} \FunctionTok{paste}\NormalTok{(}\StringTok{" Probability that exactly 38 particles }\SpecialCharTok{\textbackslash{}n}\StringTok{"}\NormalTok{, }
               \StringTok{"will be emitted during a 6.5 second }\SpecialCharTok{\textbackslash{}n}\StringTok{"}\NormalTok{, }
               \StringTok{"interval:"}\NormalTok{,}\FunctionTok{round}\NormalTok{(}\DecValTok{100}\SpecialCharTok{*}\NormalTok{result, }\DecValTok{0}\NormalTok{), }\StringTok{"\%"}\NormalTok{)}

\FunctionTok{writeLines}\NormalTok{(output)}
\end{Highlighting}
\end{Shaded}

\begin{verbatim}
##  Probability that exactly 38 particles 
##  will be emitted during a 6.5 second 
##  interval: 6 %
\end{verbatim}

\newline
\newline

\textbf{(c) {[}2 marks{]} Suppose it is known that at least 34 particles
will be emitted during this interval. What is the probability that no
more than 42 particles will be emitted during this interval?}

\begin{Shaded}
\begin{Highlighting}[]
\NormalTok{result }\OtherTok{=}\NormalTok{ (}\FunctionTok{ppois}\NormalTok{(}\DecValTok{42}\NormalTok{, }\AttributeTok{lambda=}\NormalTok{(}\FloatTok{5.6}\SpecialCharTok{*}\FloatTok{6.5}\NormalTok{)) }\SpecialCharTok{{-}} \FunctionTok{ppois}\NormalTok{(}\DecValTok{33}\NormalTok{, }\AttributeTok{lambda=}\NormalTok{(}\FloatTok{5.6}\SpecialCharTok{*}\FloatTok{6.5}\NormalTok{))) }\SpecialCharTok{/} 
\NormalTok{         (}\DecValTok{1} \SpecialCharTok{{-}} \FunctionTok{ppois}\NormalTok{(}\DecValTok{33}\NormalTok{, }\AttributeTok{lambda=}\NormalTok{(}\FloatTok{5.6}\SpecialCharTok{*}\FloatTok{6.5}\NormalTok{)))}

\NormalTok{result}
\end{Highlighting}
\end{Shaded}

\begin{verbatim}
## [1] 0.7697563
\end{verbatim}

\begin{Shaded}
\begin{Highlighting}[]
\NormalTok{output }\OtherTok{=} \FunctionTok{paste}\NormalTok{(}\StringTok{" Probability that no more than 42 particles }\SpecialCharTok{\textbackslash{}n}\StringTok{"}\NormalTok{, }
               \StringTok{"will be emitted—given that at least 34 partercles }\SpecialCharTok{\textbackslash{}n}\StringTok{"}\NormalTok{, }
               \StringTok{"will be emitted—during a 6.5 second }\SpecialCharTok{\textbackslash{}n}\StringTok{"}\NormalTok{, }
               \StringTok{"interval:"}\NormalTok{,}\FunctionTok{round}\NormalTok{(}\DecValTok{100}\SpecialCharTok{*}\NormalTok{result, }\DecValTok{0}\NormalTok{), }\StringTok{"\%"}\NormalTok{)}

\FunctionTok{writeLines}\NormalTok{(output)}
\end{Highlighting}
\end{Shaded}

\begin{verbatim}
##  Probability that no more than 42 particles 
##  will be emitted—given that at least 34 partercles 
##  will be emitted—during a 6.5 second 
##  interval: 77 %
\end{verbatim}

\newpage

\hypertarget{part-2}{%
\subsection{Part 2}\label{part-2}}

A manufacturer of ceramic blades estimates that 0.81\% of all blades
produced are too brittle to use. Suppose we take a random sample of 145
blades and test them for brittleness. We want to find the probability
that at least 3 blades will be too brittle to use.

\newline
\newline

\textbf{(a) {[}1 mark{]} Find the exact probability that at least 3
blades will be too brittle to use.}

\begin{Shaded}
\begin{Highlighting}[]
\NormalTok{result }\OtherTok{=} \DecValTok{1} \SpecialCharTok{{-}} \FunctionTok{pbinom}\NormalTok{(}\DecValTok{2}\NormalTok{, }\AttributeTok{size=}\DecValTok{145}\NormalTok{, }\AttributeTok{prob=}\FloatTok{0.0081}\NormalTok{)}

\NormalTok{result}
\end{Highlighting}
\end{Shaded}

\begin{verbatim}
## [1] 0.1143122
\end{verbatim}

\begin{Shaded}
\begin{Highlighting}[]
\NormalTok{output }\OtherTok{=} \FunctionTok{paste}\NormalTok{(}\StringTok{" Probability that at least 3 blades will be }\SpecialCharTok{\textbackslash{}n}\StringTok{"}\NormalTok{, }
               \StringTok{"too brittle to use:"}\NormalTok{,}\FunctionTok{round}\NormalTok{(}\DecValTok{100}\SpecialCharTok{*}\NormalTok{result, }\DecValTok{0}\NormalTok{), }\StringTok{"\%"}\NormalTok{)}

\FunctionTok{writeLines}\NormalTok{(output)}
\end{Highlighting}
\end{Shaded}

\begin{verbatim}
##  Probability that at least 3 blades will be 
##  too brittle to use: 11 %
\end{verbatim}

\newline
\newline

\textbf{(b) {[}1 mark{]} Use an appropriate approximation to find the
approximate probability that at least 3 blades will be too brittle to
use.}

\begin{Shaded}
\begin{Highlighting}[]
\NormalTok{output }\OtherTok{=} \FunctionTok{paste}\NormalTok{(}\StringTok{" The questions states that n=145, p=0.0081, np=1.1745.}\SpecialCharTok{\textbackslash{}n}\StringTok{"}\NormalTok{, }
               \StringTok{"Since n\textgreater{}=100, p\textless{}=100, and np\textless{}=20, Poisson Approximation}\SpecialCharTok{\textbackslash{}n}\StringTok{"}\NormalTok{, }
               \StringTok{"may be used with lambda=n*p"}\NormalTok{)}
\FunctionTok{writeLines}\NormalTok{(output)}
\end{Highlighting}
\end{Shaded}

\begin{verbatim}
##  The questions states that n=145, p=0.0081, np=1.1745.
##  Since n>=100, p<=100, and np<=20, Poisson Approximation
##  may be used with lambda=n*p
\end{verbatim}

\begin{Shaded}
\begin{Highlighting}[]
\NormalTok{result }\OtherTok{=} \DecValTok{1} \SpecialCharTok{{-}} \FunctionTok{ppois}\NormalTok{(}\DecValTok{2}\NormalTok{, }\AttributeTok{lambda=}\FloatTok{1.1745}\NormalTok{)}

\NormalTok{result}
\end{Highlighting}
\end{Shaded}

\begin{verbatim}
## [1] 0.1150305
\end{verbatim}

\begin{Shaded}
\begin{Highlighting}[]
\NormalTok{output2 }\OtherTok{=} \FunctionTok{paste}\NormalTok{(}\StringTok{" Probability that at least 3 blades will be}\SpecialCharTok{\textbackslash{}n}\StringTok{"}\NormalTok{, }
                \StringTok{"too brittle to use, using a }\SpecialCharTok{\textbackslash{}n}\StringTok{"}\NormalTok{, }
                \StringTok{"Poisson Approximation:"}\NormalTok{,}\FunctionTok{round}\NormalTok{(}\DecValTok{100}\SpecialCharTok{*}\NormalTok{result, }\DecValTok{0}\NormalTok{), }\StringTok{"\%"}\NormalTok{)}

\FunctionTok{writeLines}\NormalTok{(output2)}
\end{Highlighting}
\end{Shaded}

\begin{verbatim}
##  Probability that at least 3 blades will be
##  too brittle to use, using a 
##  Poisson Approximation: 12 %
\end{verbatim}

\newpage

\hypertarget{part-3}{%
\subsection{Part 3}\label{part-3}}

The fracture toughness (in MP a√m) of a particular steel alloy is known
to be normally distributed with a mean of 29.2 and a standard deviation
of 2.17. We select one sample of this alloy at random and measure its
fracture toughness.

\newline
\newline

\textbf{(a) {[}1 mark{]} What is the probability that the fracture
toughness will be between 24.8 and 31.5?}

\begin{Shaded}
\begin{Highlighting}[]
\NormalTok{result }\OtherTok{=} \FunctionTok{pnorm}\NormalTok{(}\FloatTok{31.5}\NormalTok{, }\AttributeTok{mean=}\FloatTok{29.2}\NormalTok{, }\AttributeTok{sd=}\FloatTok{2.17}\NormalTok{) }\SpecialCharTok{{-}} \FunctionTok{pnorm}\NormalTok{(}\FloatTok{24.8}\NormalTok{, }\AttributeTok{mean=}\FloatTok{29.2}\NormalTok{, }\AttributeTok{sd=}\FloatTok{2.17}\NormalTok{)}

\NormalTok{result}
\end{Highlighting}
\end{Shaded}

\begin{verbatim}
## [1] 0.8341087
\end{verbatim}

\begin{Shaded}
\begin{Highlighting}[]
\NormalTok{output }\OtherTok{=} \FunctionTok{paste}\NormalTok{(}\StringTok{" Probability that the fracture toughness }\SpecialCharTok{\textbackslash{}n}\StringTok{"}\NormalTok{, }
               \StringTok{"will be between 24.8 and 31.5:"}\NormalTok{,}\FunctionTok{round}\NormalTok{(}\DecValTok{100}\SpecialCharTok{*}\NormalTok{result, }\DecValTok{0}\NormalTok{), }\StringTok{"\%"}\NormalTok{)}

\FunctionTok{writeLines}\NormalTok{(output)}
\end{Highlighting}
\end{Shaded}

\begin{verbatim}
##  Probability that the fracture toughness 
##  will be between 24.8 and 31.5: 83 %
\end{verbatim}

\newline
\newline

\textbf{(b) {[}1 mark{]} What is the probability that the fracture
toughness will be at least 28.2?}

\begin{Shaded}
\begin{Highlighting}[]
\NormalTok{result }\OtherTok{=} \DecValTok{1} \SpecialCharTok{{-}} \FunctionTok{pnorm}\NormalTok{(}\FloatTok{28.2}\NormalTok{, }\AttributeTok{mean=}\FloatTok{29.2}\NormalTok{, }\AttributeTok{sd=}\FloatTok{2.17}\NormalTok{)}

\NormalTok{result}
\end{Highlighting}
\end{Shaded}

\begin{verbatim}
## [1] 0.6775395
\end{verbatim}

\begin{Shaded}
\begin{Highlighting}[]
\FunctionTok{paste}\NormalTok{(}\StringTok{"Probability that the fracture toughness will be at least 28.2:"}\NormalTok{,}\FunctionTok{round}\NormalTok{(}\DecValTok{100}\SpecialCharTok{*}\NormalTok{result, }\DecValTok{0}\NormalTok{), }\StringTok{"\%"}\NormalTok{)}
\end{Highlighting}
\end{Shaded}

\begin{verbatim}
## [1] "Probability that the fracture toughness will be at least 28.2: 68 %"
\end{verbatim}

\newline
\newline

\textbf{(c) {[}2 marks{]} Given that the fracture toughness is at least
26, what is the probability that the fracture toughness will be no more
than 32.1?}

\begin{Shaded}
\begin{Highlighting}[]
\NormalTok{result }\OtherTok{=}\NormalTok{ (}\FunctionTok{pnorm}\NormalTok{(}\FloatTok{32.1}\NormalTok{, }\AttributeTok{mean=}\FloatTok{29.2}\NormalTok{, }\AttributeTok{sd=}\FloatTok{2.17}\NormalTok{) }\SpecialCharTok{{-}} \FunctionTok{pnorm}\NormalTok{(}\DecValTok{26}\NormalTok{, }\AttributeTok{mean=}\FloatTok{29.2}\NormalTok{, }\AttributeTok{sd=}\FloatTok{2.17}\NormalTok{)) }\SpecialCharTok{/} 
\NormalTok{         (}\DecValTok{1} \SpecialCharTok{{-}} \FunctionTok{pnorm}\NormalTok{(}\DecValTok{26}\NormalTok{, }\AttributeTok{mean=}\FloatTok{29.2}\NormalTok{, }\AttributeTok{sd=}\FloatTok{2.17}\NormalTok{))}

\NormalTok{result}
\end{Highlighting}
\end{Shaded}

\begin{verbatim}
## [1] 0.9024481
\end{verbatim}

\begin{Shaded}
\begin{Highlighting}[]
\NormalTok{output }\OtherTok{=} \FunctionTok{paste}\NormalTok{(}\StringTok{" Probability that the fracture toughness will be no more than}\SpecialCharTok{\textbackslash{}n}\StringTok{"}\NormalTok{, }
               \StringTok{"32.1 given that it is at least 26:"}\NormalTok{,}\FunctionTok{round}\NormalTok{(}\DecValTok{100}\SpecialCharTok{*}\NormalTok{result, }\DecValTok{0}\NormalTok{), }\StringTok{"\%"}\NormalTok{)}
\end{Highlighting}
\end{Shaded}

\newpage

\hypertarget{part-4}{%
\subsection{Part 4}\label{part-4}}

The purpose of this question is to help you visualize the normal
approximation to the binomial distribution which have seen in Set 16.

\newline
\newline

\textbf{(a) {[}1 mark{]} Let X ∼ binomial(n = 85, p = 0.32). Create a
vector called simulation.data which contains a simulation for 3700
values for X. (i.e.~Simulate 3700 experiments, each being binomial with
n = 85 and p = 0.32.) Provide a copy of the R command which you used to
create this vector. You do not need to copy the 3700 values you
generated.}

\begin{Shaded}
\begin{Highlighting}[]
\NormalTok{simulation.data }\OtherTok{=} \FunctionTok{rbinom}\NormalTok{(}\DecValTok{3700}\NormalTok{, }\AttributeTok{size=}\DecValTok{84}\NormalTok{, }\AttributeTok{prob=}\FloatTok{0.32}\NormalTok{)}
\end{Highlighting}
\end{Shaded}

\newline
\newline

\textbf{(b) {[}2 marks{]} Create a histogram of simulation.data and copy
it and your line of R code into your assignment. Your histogram should
have an appropriate title and an appropriate label on the x-axis.
Comment on the shape of the histogram. (We are looking for a single
phrase here to describe the histogram. It should be a shape we've
discussed recently.)}

\begin{Shaded}
\begin{Highlighting}[]
\FunctionTok{hist}\NormalTok{(simulation.data,}
     \AttributeTok{main=}\StringTok{"Sumulation of X\textasciitilde{}binomial(n=85, p=0.32) * 3700"}\NormalTok{,}
     \AttributeTok{xlab=}\StringTok{"Individual Binomial Dist. Values"}\NormalTok{)}
\end{Highlighting}
\end{Shaded}

\includegraphics{R-Assignment-2_files/figure-latex/4b-1.pdf}

\begin{Shaded}
\begin{Highlighting}[]
\NormalTok{output }\OtherTok{=} \FunctionTok{paste}\NormalTok{(}\StringTok{"The shape of the histogram appears to be a bell}\SpecialCharTok{\textbackslash{}n}\StringTok{"}\NormalTok{, }
               \StringTok{"curve and closely resembles the normal distribution."}\NormalTok{)}
\FunctionTok{writeLines}\NormalTok{(output)}
\end{Highlighting}
\end{Shaded}

\begin{verbatim}
## The shape of the histogram appears to be a bell
##  curve and closely resembles the normal distribution.
\end{verbatim}

\newline
\newline

\textbf{(c) {[}2 marks{]} Calculate the sample mean of simulation.data.
Copy the command used, and the output. How close is your sample mean to
what you would expect? (Hint: We have discussed the expected value of
the sample mean X. We have also discussed the expected value of a
binomial random variable X.)}

\begin{Shaded}
\begin{Highlighting}[]
\NormalTok{result.mean }\OtherTok{=} \FunctionTok{mean}\NormalTok{(simulation.data)}
\FunctionTok{paste}\NormalTok{(}\StringTok{"Mean of 3700 experiments:"}\NormalTok{, }\FunctionTok{round}\NormalTok{(result.mean, }\DecValTok{1}\NormalTok{))}
\end{Highlighting}
\end{Shaded}

\begin{verbatim}
## [1] "Mean of 3700 experiments: 26.9"
\end{verbatim}

\begin{Shaded}
\begin{Highlighting}[]
\NormalTok{result.np }\OtherTok{=} \DecValTok{85}\SpecialCharTok{*}\FloatTok{0.32}
\FunctionTok{paste}\NormalTok{(}\StringTok{"Expected value of n*p:"}\NormalTok{,}
\NormalTok{result.np)}
\end{Highlighting}
\end{Shaded}

\begin{verbatim}
## [1] "Expected value of n*p: 27.2"
\end{verbatim}

\begin{Shaded}
\begin{Highlighting}[]
\FunctionTok{print}\NormalTok{(}\StringTok{"Calculating the mean the binomial distribution experiments gives us an answer remarkably close to our formula for the expected value of binomial distributions. It is within 1\% of accuracy it seems."}\NormalTok{)}
\end{Highlighting}
\end{Shaded}

\begin{verbatim}
## [1] "Calculating the mean the binomial distribution experiments gives us an answer remarkably close to our formula for the expected value of binomial distributions. It is within 1% of accuracy it seems."
\end{verbatim}

\end{document}
